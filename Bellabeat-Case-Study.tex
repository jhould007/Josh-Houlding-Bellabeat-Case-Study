% Options for packages loaded elsewhere
\PassOptionsToPackage{unicode}{hyperref}
\PassOptionsToPackage{hyphens}{url}
%
\documentclass[
]{article}
\usepackage{amsmath,amssymb}
\usepackage{iftex}
\ifPDFTeX
  \usepackage[T1]{fontenc}
  \usepackage[utf8]{inputenc}
  \usepackage{textcomp} % provide euro and other symbols
\else % if luatex or xetex
  \usepackage{unicode-math} % this also loads fontspec
  \defaultfontfeatures{Scale=MatchLowercase}
  \defaultfontfeatures[\rmfamily]{Ligatures=TeX,Scale=1}
\fi
\usepackage{lmodern}
\ifPDFTeX\else
  % xetex/luatex font selection
\fi
% Use upquote if available, for straight quotes in verbatim environments
\IfFileExists{upquote.sty}{\usepackage{upquote}}{}
\IfFileExists{microtype.sty}{% use microtype if available
  \usepackage[]{microtype}
  \UseMicrotypeSet[protrusion]{basicmath} % disable protrusion for tt fonts
}{}
\makeatletter
\@ifundefined{KOMAClassName}{% if non-KOMA class
  \IfFileExists{parskip.sty}{%
    \usepackage{parskip}
  }{% else
    \setlength{\parindent}{0pt}
    \setlength{\parskip}{6pt plus 2pt minus 1pt}}
}{% if KOMA class
  \KOMAoptions{parskip=half}}
\makeatother
\usepackage{xcolor}
\usepackage[margin=1in]{geometry}
\usepackage{color}
\usepackage{fancyvrb}
\newcommand{\VerbBar}{|}
\newcommand{\VERB}{\Verb[commandchars=\\\{\}]}
\DefineVerbatimEnvironment{Highlighting}{Verbatim}{commandchars=\\\{\}}
% Add ',fontsize=\small' for more characters per line
\usepackage{framed}
\definecolor{shadecolor}{RGB}{248,248,248}
\newenvironment{Shaded}{\begin{snugshade}}{\end{snugshade}}
\newcommand{\AlertTok}[1]{\textcolor[rgb]{0.94,0.16,0.16}{#1}}
\newcommand{\AnnotationTok}[1]{\textcolor[rgb]{0.56,0.35,0.01}{\textbf{\textit{#1}}}}
\newcommand{\AttributeTok}[1]{\textcolor[rgb]{0.13,0.29,0.53}{#1}}
\newcommand{\BaseNTok}[1]{\textcolor[rgb]{0.00,0.00,0.81}{#1}}
\newcommand{\BuiltInTok}[1]{#1}
\newcommand{\CharTok}[1]{\textcolor[rgb]{0.31,0.60,0.02}{#1}}
\newcommand{\CommentTok}[1]{\textcolor[rgb]{0.56,0.35,0.01}{\textit{#1}}}
\newcommand{\CommentVarTok}[1]{\textcolor[rgb]{0.56,0.35,0.01}{\textbf{\textit{#1}}}}
\newcommand{\ConstantTok}[1]{\textcolor[rgb]{0.56,0.35,0.01}{#1}}
\newcommand{\ControlFlowTok}[1]{\textcolor[rgb]{0.13,0.29,0.53}{\textbf{#1}}}
\newcommand{\DataTypeTok}[1]{\textcolor[rgb]{0.13,0.29,0.53}{#1}}
\newcommand{\DecValTok}[1]{\textcolor[rgb]{0.00,0.00,0.81}{#1}}
\newcommand{\DocumentationTok}[1]{\textcolor[rgb]{0.56,0.35,0.01}{\textbf{\textit{#1}}}}
\newcommand{\ErrorTok}[1]{\textcolor[rgb]{0.64,0.00,0.00}{\textbf{#1}}}
\newcommand{\ExtensionTok}[1]{#1}
\newcommand{\FloatTok}[1]{\textcolor[rgb]{0.00,0.00,0.81}{#1}}
\newcommand{\FunctionTok}[1]{\textcolor[rgb]{0.13,0.29,0.53}{\textbf{#1}}}
\newcommand{\ImportTok}[1]{#1}
\newcommand{\InformationTok}[1]{\textcolor[rgb]{0.56,0.35,0.01}{\textbf{\textit{#1}}}}
\newcommand{\KeywordTok}[1]{\textcolor[rgb]{0.13,0.29,0.53}{\textbf{#1}}}
\newcommand{\NormalTok}[1]{#1}
\newcommand{\OperatorTok}[1]{\textcolor[rgb]{0.81,0.36,0.00}{\textbf{#1}}}
\newcommand{\OtherTok}[1]{\textcolor[rgb]{0.56,0.35,0.01}{#1}}
\newcommand{\PreprocessorTok}[1]{\textcolor[rgb]{0.56,0.35,0.01}{\textit{#1}}}
\newcommand{\RegionMarkerTok}[1]{#1}
\newcommand{\SpecialCharTok}[1]{\textcolor[rgb]{0.81,0.36,0.00}{\textbf{#1}}}
\newcommand{\SpecialStringTok}[1]{\textcolor[rgb]{0.31,0.60,0.02}{#1}}
\newcommand{\StringTok}[1]{\textcolor[rgb]{0.31,0.60,0.02}{#1}}
\newcommand{\VariableTok}[1]{\textcolor[rgb]{0.00,0.00,0.00}{#1}}
\newcommand{\VerbatimStringTok}[1]{\textcolor[rgb]{0.31,0.60,0.02}{#1}}
\newcommand{\WarningTok}[1]{\textcolor[rgb]{0.56,0.35,0.01}{\textbf{\textit{#1}}}}
\usepackage{longtable,booktabs,array}
\usepackage{calc} % for calculating minipage widths
% Correct order of tables after \paragraph or \subparagraph
\usepackage{etoolbox}
\makeatletter
\patchcmd\longtable{\par}{\if@noskipsec\mbox{}\fi\par}{}{}
\makeatother
% Allow footnotes in longtable head/foot
\IfFileExists{footnotehyper.sty}{\usepackage{footnotehyper}}{\usepackage{footnote}}
\makesavenoteenv{longtable}
\usepackage{graphicx}
\makeatletter
\def\maxwidth{\ifdim\Gin@nat@width>\linewidth\linewidth\else\Gin@nat@width\fi}
\def\maxheight{\ifdim\Gin@nat@height>\textheight\textheight\else\Gin@nat@height\fi}
\makeatother
% Scale images if necessary, so that they will not overflow the page
% margins by default, and it is still possible to overwrite the defaults
% using explicit options in \includegraphics[width, height, ...]{}
\setkeys{Gin}{width=\maxwidth,height=\maxheight,keepaspectratio}
% Set default figure placement to htbp
\makeatletter
\def\fps@figure{htbp}
\makeatother
\setlength{\emergencystretch}{3em} % prevent overfull lines
\providecommand{\tightlist}{%
  \setlength{\itemsep}{0pt}\setlength{\parskip}{0pt}}
\setcounter{secnumdepth}{-\maxdimen} % remove section numbering
\ifLuaTeX
  \usepackage{selnolig}  % disable illegal ligatures
\fi
\IfFileExists{bookmark.sty}{\usepackage{bookmark}}{\usepackage{hyperref}}
\IfFileExists{xurl.sty}{\usepackage{xurl}}{} % add URL line breaks if available
\urlstyle{same}
\hypersetup{
  pdftitle={Bellabeat Case Study},
  pdfauthor={Josh Houlding},
  hidelinks,
  pdfcreator={LaTeX via pandoc}}

\title{Bellabeat Case Study}
\author{Josh Houlding}
\date{2023-07-30}

\begin{document}
\maketitle

\hypertarget{about-bellabeat}{%
\section{About Bellabeat}\label{about-bellabeat}}

Bellabeat is a technology company founded by Urška Sršen and Sandro Mur
that produces wearable devices targeted toward women. These devices are
intended to provide women with data on activity, sleep, stress, and
reproductive health, empowering them to make healthier life decisions.
According to their website, a ``massive majority of all products are
created for and tested on men, then marketed to women.'' This means they
have a good chance of being sub-optimal for women who buy them. As such,
Bellabeat's mission is to provide women with devices made for them that
can track menstrual cycles, as they play a huge role in a woman's
health.

\hypertarget{the-objective}{%
\section{The objective}\label{the-objective}}

Sršen believes that by analyzing usage data from non-Bellabeat smart
devices to understand how consumers use them, the marketing analytics
division can come up with recommendations to direct future Bellabeat
marketing strategies.

\hypertarget{the-data}{%
\section{The data}\label{the-data}}

This analysis was done with the
\href{https://www.kaggle.com/datasets/arashnic/fitbit}{FitBit Fitness
Tracker Data} dataset, publicly available on the data-sharing website
known as Kaggle.According to the Kaggle page, it consists of personal
tracker data from 30 eligible and consenting Fitbit users shared via a
distributed survey in Amazon Mechanical Turk, consisting of
minute-to-minute physical activity, heart rate, steps, distance, and
sleep monitoring.

The dataset has 18 CSV files, all containing various combinations of
these attributes. Because many of the files contain duplicate data, I
chose the following 3 files which were the most relevant to my analysis:

\begin{Shaded}
\begin{Highlighting}[]
\FunctionTok{library}\NormalTok{(tidyverse)}
\FunctionTok{library}\NormalTok{(knitr)}
\NormalTok{dailyActivity }\OtherTok{\textless{}{-}} \FunctionTok{read\_csv}\NormalTok{(}\StringTok{"dailyActivity\_merged.csv"}\NormalTok{)}
\FunctionTok{kable}\NormalTok{(}\FunctionTok{head}\NormalTok{(dailyActivity), }\AttributeTok{caption =} \StringTok{"dailyActivity\_merged.csv."}\NormalTok{)}
\end{Highlighting}
\end{Shaded}

\begin{longtable}[]{@{}
  >{\raggedleft\arraybackslash}p{(\columnwidth - 28\tabcolsep) * \real{0.0418}}
  >{\raggedright\arraybackslash}p{(\columnwidth - 28\tabcolsep) * \real{0.0494}}
  >{\raggedleft\arraybackslash}p{(\columnwidth - 28\tabcolsep) * \real{0.0418}}
  >{\raggedleft\arraybackslash}p{(\columnwidth - 28\tabcolsep) * \real{0.0532}}
  >{\raggedleft\arraybackslash}p{(\columnwidth - 28\tabcolsep) * \real{0.0608}}
  >{\raggedleft\arraybackslash}p{(\columnwidth - 28\tabcolsep) * \real{0.0951}}
  >{\raggedleft\arraybackslash}p{(\columnwidth - 28\tabcolsep) * \real{0.0722}}
  >{\raggedleft\arraybackslash}p{(\columnwidth - 28\tabcolsep) * \real{0.0951}}
  >{\raggedleft\arraybackslash}p{(\columnwidth - 28\tabcolsep) * \real{0.0760}}
  >{\raggedleft\arraybackslash}p{(\columnwidth - 28\tabcolsep) * \real{0.0913}}
  >{\raggedleft\arraybackslash}p{(\columnwidth - 28\tabcolsep) * \real{0.0684}}
  >{\raggedleft\arraybackslash}p{(\columnwidth - 28\tabcolsep) * \real{0.0760}}
  >{\raggedleft\arraybackslash}p{(\columnwidth - 28\tabcolsep) * \real{0.0798}}
  >{\raggedleft\arraybackslash}p{(\columnwidth - 28\tabcolsep) * \real{0.0646}}
  >{\raggedleft\arraybackslash}p{(\columnwidth - 28\tabcolsep) * \real{0.0342}}@{}}
\caption{dailyActivity\_merged.csv.}\tabularnewline
\toprule\noalign{}
\begin{minipage}[b]{\linewidth}\raggedleft
Id
\end{minipage} & \begin{minipage}[b]{\linewidth}\raggedright
ActivityDate
\end{minipage} & \begin{minipage}[b]{\linewidth}\raggedleft
TotalSteps
\end{minipage} & \begin{minipage}[b]{\linewidth}\raggedleft
TotalDistance
\end{minipage} & \begin{minipage}[b]{\linewidth}\raggedleft
TrackerDistance
\end{minipage} & \begin{minipage}[b]{\linewidth}\raggedleft
LoggedActivitiesDistance
\end{minipage} & \begin{minipage}[b]{\linewidth}\raggedleft
VeryActiveDistance
\end{minipage} & \begin{minipage}[b]{\linewidth}\raggedleft
ModeratelyActiveDistance
\end{minipage} & \begin{minipage}[b]{\linewidth}\raggedleft
LightActiveDistance
\end{minipage} & \begin{minipage}[b]{\linewidth}\raggedleft
SedentaryActiveDistance
\end{minipage} & \begin{minipage}[b]{\linewidth}\raggedleft
VeryActiveMinutes
\end{minipage} & \begin{minipage}[b]{\linewidth}\raggedleft
FairlyActiveMinutes
\end{minipage} & \begin{minipage}[b]{\linewidth}\raggedleft
LightlyActiveMinutes
\end{minipage} & \begin{minipage}[b]{\linewidth}\raggedleft
SedentaryMinutes
\end{minipage} & \begin{minipage}[b]{\linewidth}\raggedleft
Calories
\end{minipage} \\
\midrule\noalign{}
\endfirsthead
\toprule\noalign{}
\begin{minipage}[b]{\linewidth}\raggedleft
Id
\end{minipage} & \begin{minipage}[b]{\linewidth}\raggedright
ActivityDate
\end{minipage} & \begin{minipage}[b]{\linewidth}\raggedleft
TotalSteps
\end{minipage} & \begin{minipage}[b]{\linewidth}\raggedleft
TotalDistance
\end{minipage} & \begin{minipage}[b]{\linewidth}\raggedleft
TrackerDistance
\end{minipage} & \begin{minipage}[b]{\linewidth}\raggedleft
LoggedActivitiesDistance
\end{minipage} & \begin{minipage}[b]{\linewidth}\raggedleft
VeryActiveDistance
\end{minipage} & \begin{minipage}[b]{\linewidth}\raggedleft
ModeratelyActiveDistance
\end{minipage} & \begin{minipage}[b]{\linewidth}\raggedleft
LightActiveDistance
\end{minipage} & \begin{minipage}[b]{\linewidth}\raggedleft
SedentaryActiveDistance
\end{minipage} & \begin{minipage}[b]{\linewidth}\raggedleft
VeryActiveMinutes
\end{minipage} & \begin{minipage}[b]{\linewidth}\raggedleft
FairlyActiveMinutes
\end{minipage} & \begin{minipage}[b]{\linewidth}\raggedleft
LightlyActiveMinutes
\end{minipage} & \begin{minipage}[b]{\linewidth}\raggedleft
SedentaryMinutes
\end{minipage} & \begin{minipage}[b]{\linewidth}\raggedleft
Calories
\end{minipage} \\
\midrule\noalign{}
\endhead
\bottomrule\noalign{}
\endlastfoot
1503960366 & 4/12/2016 & 13162 & 8.50 & 8.50 & 0 & 1.88 & 0.55 & 6.06 &
0 & 25 & 13 & 328 & 728 & 1985 \\
1503960366 & 4/13/2016 & 10735 & 6.97 & 6.97 & 0 & 1.57 & 0.69 & 4.71 &
0 & 21 & 19 & 217 & 776 & 1797 \\
1503960366 & 4/14/2016 & 10460 & 6.74 & 6.74 & 0 & 2.44 & 0.40 & 3.91 &
0 & 30 & 11 & 181 & 1218 & 1776 \\
1503960366 & 4/15/2016 & 9762 & 6.28 & 6.28 & 0 & 2.14 & 1.26 & 2.83 & 0
& 29 & 34 & 209 & 726 & 1745 \\
1503960366 & 4/16/2016 & 12669 & 8.16 & 8.16 & 0 & 2.71 & 0.41 & 5.04 &
0 & 36 & 10 & 221 & 773 & 1863 \\
1503960366 & 4/17/2016 & 9705 & 6.48 & 6.48 & 0 & 3.19 & 0.78 & 2.51 & 0
& 38 & 20 & 164 & 539 & 1728 \\
\end{longtable}

\begin{Shaded}
\begin{Highlighting}[]
\NormalTok{sleepDay }\OtherTok{\textless{}{-}} \FunctionTok{read\_csv}\NormalTok{(}\StringTok{"sleepDay\_merged.csv"}\NormalTok{)}
\FunctionTok{kable}\NormalTok{(}\FunctionTok{head}\NormalTok{(sleepDay), }\AttributeTok{caption =} \StringTok{"sleepDay\_merged.csv."}\NormalTok{)}
\end{Highlighting}
\end{Shaded}

\begin{longtable}[]{@{}
  >{\raggedleft\arraybackslash}p{(\columnwidth - 8\tabcolsep) * \real{0.1294}}
  >{\raggedright\arraybackslash}p{(\columnwidth - 8\tabcolsep) * \real{0.2588}}
  >{\raggedleft\arraybackslash}p{(\columnwidth - 8\tabcolsep) * \real{0.2118}}
  >{\raggedleft\arraybackslash}p{(\columnwidth - 8\tabcolsep) * \real{0.2235}}
  >{\raggedleft\arraybackslash}p{(\columnwidth - 8\tabcolsep) * \real{0.1765}}@{}}
\caption{sleepDay\_merged.csv.}\tabularnewline
\toprule\noalign{}
\begin{minipage}[b]{\linewidth}\raggedleft
Id
\end{minipage} & \begin{minipage}[b]{\linewidth}\raggedright
SleepDay
\end{minipage} & \begin{minipage}[b]{\linewidth}\raggedleft
TotalSleepRecords
\end{minipage} & \begin{minipage}[b]{\linewidth}\raggedleft
TotalMinutesAsleep
\end{minipage} & \begin{minipage}[b]{\linewidth}\raggedleft
TotalTimeInBed
\end{minipage} \\
\midrule\noalign{}
\endfirsthead
\toprule\noalign{}
\begin{minipage}[b]{\linewidth}\raggedleft
Id
\end{minipage} & \begin{minipage}[b]{\linewidth}\raggedright
SleepDay
\end{minipage} & \begin{minipage}[b]{\linewidth}\raggedleft
TotalSleepRecords
\end{minipage} & \begin{minipage}[b]{\linewidth}\raggedleft
TotalMinutesAsleep
\end{minipage} & \begin{minipage}[b]{\linewidth}\raggedleft
TotalTimeInBed
\end{minipage} \\
\midrule\noalign{}
\endhead
\bottomrule\noalign{}
\endlastfoot
1503960366 & 4/12/2016 12:00:00 AM & 1 & 327 & 346 \\
1503960366 & 4/13/2016 12:00:00 AM & 2 & 384 & 407 \\
1503960366 & 4/15/2016 12:00:00 AM & 1 & 412 & 442 \\
1503960366 & 4/16/2016 12:00:00 AM & 2 & 340 & 367 \\
1503960366 & 4/17/2016 12:00:00 AM & 1 & 700 & 712 \\
1503960366 & 4/19/2016 12:00:00 AM & 1 & 304 & 320 \\
\end{longtable}

\begin{Shaded}
\begin{Highlighting}[]
\NormalTok{weightLogInfo }\OtherTok{\textless{}{-}} \FunctionTok{read\_csv}\NormalTok{(}\StringTok{"weightLogInfo\_merged.csv"}\NormalTok{)}
\FunctionTok{kable}\NormalTok{(}\FunctionTok{head}\NormalTok{(weightLogInfo), }\AttributeTok{caption =} \StringTok{"weightLogInfo\_merged.csv."}\NormalTok{)}
\end{Highlighting}
\end{Shaded}

\begin{longtable}[]{@{}
  >{\raggedleft\arraybackslash}p{(\columnwidth - 14\tabcolsep) * \real{0.1183}}
  >{\raggedright\arraybackslash}p{(\columnwidth - 14\tabcolsep) * \real{0.2366}}
  >{\raggedleft\arraybackslash}p{(\columnwidth - 14\tabcolsep) * \real{0.0968}}
  >{\raggedleft\arraybackslash}p{(\columnwidth - 14\tabcolsep) * \real{0.1398}}
  >{\raggedleft\arraybackslash}p{(\columnwidth - 14\tabcolsep) * \real{0.0430}}
  >{\raggedleft\arraybackslash}p{(\columnwidth - 14\tabcolsep) * \real{0.0645}}
  >{\raggedright\arraybackslash}p{(\columnwidth - 14\tabcolsep) * \real{0.1613}}
  >{\raggedleft\arraybackslash}p{(\columnwidth - 14\tabcolsep) * \real{0.1398}}@{}}
\caption{weightLogInfo\_merged.csv.}\tabularnewline
\toprule\noalign{}
\begin{minipage}[b]{\linewidth}\raggedleft
Id
\end{minipage} & \begin{minipage}[b]{\linewidth}\raggedright
Date
\end{minipage} & \begin{minipage}[b]{\linewidth}\raggedleft
WeightKg
\end{minipage} & \begin{minipage}[b]{\linewidth}\raggedleft
WeightPounds
\end{minipage} & \begin{minipage}[b]{\linewidth}\raggedleft
Fat
\end{minipage} & \begin{minipage}[b]{\linewidth}\raggedleft
BMI
\end{minipage} & \begin{minipage}[b]{\linewidth}\raggedright
IsManualReport
\end{minipage} & \begin{minipage}[b]{\linewidth}\raggedleft
LogId
\end{minipage} \\
\midrule\noalign{}
\endfirsthead
\toprule\noalign{}
\begin{minipage}[b]{\linewidth}\raggedleft
Id
\end{minipage} & \begin{minipage}[b]{\linewidth}\raggedright
Date
\end{minipage} & \begin{minipage}[b]{\linewidth}\raggedleft
WeightKg
\end{minipage} & \begin{minipage}[b]{\linewidth}\raggedleft
WeightPounds
\end{minipage} & \begin{minipage}[b]{\linewidth}\raggedleft
Fat
\end{minipage} & \begin{minipage}[b]{\linewidth}\raggedleft
BMI
\end{minipage} & \begin{minipage}[b]{\linewidth}\raggedright
IsManualReport
\end{minipage} & \begin{minipage}[b]{\linewidth}\raggedleft
LogId
\end{minipage} \\
\midrule\noalign{}
\endhead
\bottomrule\noalign{}
\endlastfoot
1503960366 & 5/2/2016 11:59:59 PM & 52.6 & 115.9631 & 22 & 22.65 & TRUE
& 1.462234e+12 \\
1503960366 & 5/3/2016 11:59:59 PM & 52.6 & 115.9631 & NA & 22.65 & TRUE
& 1.462320e+12 \\
1927972279 & 4/13/2016 1:08:52 AM & 133.5 & 294.3171 & NA & 47.54 &
FALSE & 1.460510e+12 \\
2873212765 & 4/21/2016 11:59:59 PM & 56.7 & 125.0021 & NA & 21.45 & TRUE
& 1.461283e+12 \\
2873212765 & 5/12/2016 11:59:59 PM & 57.3 & 126.3249 & NA & 21.69 & TRUE
& 1.463098e+12 \\
4319703577 & 4/17/2016 11:59:59 PM & 72.4 & 159.6147 & 25 & 27.45 & TRUE
& 1.460938e+12 \\
\end{longtable}

\hypertarget{limitations-of-the-data}{%
\subsection{Limitations of the data}\label{limitations-of-the-data}}

\begin{itemize}
\tightlist
\item
  Demographic info on the 30 users is not available, so many of the
  users may be men. This isn't ideal when analyzing data for insights on
  women's health products.
\item
  The data was generated in a short time frame (April 12th, 2016 to May
  12th, 2016).
\item
  Users gathered their health data using a variety of different Fitbit
  trackers, each of which vary in features, adding an additional degree
  of variability to the data.
\end{itemize}

\hypertarget{cleaning-the-data}{%
\subsection{Cleaning the data}\label{cleaning-the-data}}

I took several steps to ensure the data was clean and in a usable
format.

\hypertarget{removing-entries-with-missing-values}{%
\subsubsection{Removing entries with missing
values}\label{removing-entries-with-missing-values}}

I tried running the following code to accomplish this:

\begin{Shaded}
\begin{Highlighting}[]
\NormalTok{dailyActivity }\OtherTok{\textless{}{-}} \FunctionTok{na.omit}\NormalTok{(dailyActivity)}
\NormalTok{sleepDay }\OtherTok{\textless{}{-}} \FunctionTok{na.omit}\NormalTok{(sleepDay)}
\NormalTok{weightLogInfo }\OtherTok{\textless{}{-}} \FunctionTok{na.omit}\NormalTok{(weightLogInfo)}
\end{Highlighting}
\end{Shaded}

However, attempting to remove entries with missing values from the
\texttt{weightLogInfo} was problematic because 65 out of 67 entries had
a missing fat value. This was presumably because users did not record
their fat content when weighing themselves.

The first two dataframes, \texttt{dailyActivity} and \texttt{sleepDay}
were unaffected because they did not have any missing values.

\hypertarget{removing-duplicate-entries}{%
\subsubsection{Removing duplicate
entries}\label{removing-duplicate-entries}}

I ran the following code to remove duplicate entries:

\begin{Shaded}
\begin{Highlighting}[]
\NormalTok{dailyActivity }\OtherTok{\textless{}{-}} \FunctionTok{distinct}\NormalTok{(dailyActivity)}
\NormalTok{sleepDay }\OtherTok{\textless{}{-}} \FunctionTok{distinct}\NormalTok{(sleepDay)}
\NormalTok{weightLogInfo }\OtherTok{\textless{}{-}} \FunctionTok{distinct}\NormalTok{(weightLogInfo)}
\end{Highlighting}
\end{Shaded}

\texttt{dailyActivity} and \texttt{weightLogInfo} were unaffected, but
sleepDay lost 3 duplicate entries.

\hypertarget{converting-activity-dates-from-string-text-to-date-format}{%
\subsubsection{Converting activity dates from string (text) to date
format}\label{converting-activity-dates-from-string-text-to-date-format}}

I used the following code:
\texttt{dailyActivity\$ActivityDate\ \textless{}-\ as.Date(dailyActivity\$ActivityDate)},
but the dates got messed up so I didn't continue with it.

\hypertarget{converting-id-from-numerical-values-to-strings}{%
\subsubsection{Converting Id from numerical values to
strings}\label{converting-id-from-numerical-values-to-strings}}

I thought turning the numerical Id values to strings would ensure that
they are not treated as numerical values on the x-axis when plotting
data later, so I ran the following code:

\begin{Shaded}
\begin{Highlighting}[]
\NormalTok{dailyActivity}\SpecialCharTok{$}\NormalTok{Id }\OtherTok{\textless{}{-}} \FunctionTok{as.character}\NormalTok{(dailyActivity}\SpecialCharTok{$}\NormalTok{Id)}
\NormalTok{sleepDay}\SpecialCharTok{$}\NormalTok{Id }\OtherTok{\textless{}{-}} \FunctionTok{as.character}\NormalTok{(sleepDay}\SpecialCharTok{$}\NormalTok{Id)}
\NormalTok{weightLogInfo}\SpecialCharTok{$}\NormalTok{Id }\OtherTok{\textless{}{-}} \FunctionTok{as.character}\NormalTok{(weightLogInfo}\SpecialCharTok{$}\NormalTok{Id)}
\end{Highlighting}
\end{Shaded}

\hypertarget{removing-the-120000-am-timestamp-from-the-sleepday-variable-in-the-sleepday-dataframe}{%
\subsubsection{\texorpdfstring{Removing the ``12:00:00 AM'' timestamp
from the \texttt{SleepDay} variable in the \texttt{sleepDay}
dataframe}{Removing the ``12:00:00 AM'' timestamp from the SleepDay variable in the sleepDay dataframe}}\label{removing-the-120000-am-timestamp-from-the-sleepday-variable-in-the-sleepday-dataframe}}

This was redundant since we all know days start at 12:00 AM, so I ran
this code to fix it:

\begin{Shaded}
\begin{Highlighting}[]
\NormalTok{sleepDay}\SpecialCharTok{$}\NormalTok{SleepDay }\OtherTok{\textless{}{-}} \FunctionTok{substr}\NormalTok{(sleepDay}\SpecialCharTok{$}\NormalTok{SleepDay, }\AttributeTok{start=}\DecValTok{1}\NormalTok{, }\AttributeTok{stop=}\DecValTok{9}\NormalTok{)}
\end{Highlighting}
\end{Shaded}

\hypertarget{the-analysis}{%
\section{The analysis}\label{the-analysis}}

In my analysis, there were several questions I wanted to answer, which I
will illustrate below.

\hypertarget{how-many-users-wore-their-fitbit-device-during-the-day-on-each-day}{%
\subsubsection{How many users wore their Fitbit device during the day on
each
day?}\label{how-many-users-wore-their-fitbit-device-during-the-day-on-each-day}}

I found that on every single day of data collection, more than half of
the 33 users wore their device at some point in the day.

\begin{Shaded}
\begin{Highlighting}[]
\FunctionTok{library}\NormalTok{(dplyr)}
\FunctionTok{library}\NormalTok{(ggplot2)}

\CommentTok{\# Filter out rows with 0 steps}
\NormalTok{filtered\_data }\OtherTok{\textless{}{-}}\NormalTok{ dailyActivity }\SpecialCharTok{\%\textgreater{}\%}
  \FunctionTok{filter}\NormalTok{(TotalSteps }\SpecialCharTok{\textgreater{}} \DecValTok{0}\NormalTok{)}

\CommentTok{\# Group data by date and count the number of unique users}
\NormalTok{step\_counts\_by\_date }\OtherTok{\textless{}{-}}\NormalTok{ filtered\_data }\SpecialCharTok{\%\textgreater{}\%}
  \FunctionTok{group\_by}\NormalTok{(ActivityDate) }\SpecialCharTok{\%\textgreater{}\%}
  \FunctionTok{summarise}\NormalTok{(}\AttributeTok{num\_users =} \FunctionTok{n\_distinct}\NormalTok{(Id))}

\CommentTok{\# Find what \% of the time people were wearing their Fitbits overall}
\NormalTok{avg\_users\_per\_day }\OtherTok{\textless{}{-}} \FunctionTok{sum}\NormalTok{(step\_counts\_by\_date}\SpecialCharTok{$}\NormalTok{num\_users)}\SpecialCharTok{/}\DecValTok{31}
\NormalTok{overall\_percent }\OtherTok{\textless{}{-}}\NormalTok{ (avg\_users\_per\_day}\SpecialCharTok{/}\DecValTok{31}\NormalTok{)}\SpecialCharTok{*}\DecValTok{100} 
\FunctionTok{print}\NormalTok{(overall\_percent)}
\end{Highlighting}
\end{Shaded}

\begin{verbatim}
## [1] 89.80229
\end{verbatim}

\begin{Shaded}
\begin{Highlighting}[]
\CommentTok{\# Create the bar plot}
\FunctionTok{ggplot}\NormalTok{(step\_counts\_by\_date, }\FunctionTok{aes}\NormalTok{(}\AttributeTok{x =} \FunctionTok{reorder}\NormalTok{(ActivityDate, num\_users))) }\SpecialCharTok{+}
  \FunctionTok{geom\_bar}\NormalTok{(}\AttributeTok{mapping=}\FunctionTok{aes}\NormalTok{(}\AttributeTok{y=}\NormalTok{num\_users), }\AttributeTok{stat =} \StringTok{"identity"}\NormalTok{, }\AttributeTok{fill =} \StringTok{"blue"}\NormalTok{, }\AttributeTok{alpha =} \FloatTok{0.7}\NormalTok{) }\SpecialCharTok{+} 
  \FunctionTok{annotate}\NormalTok{(}\AttributeTok{geom=}\StringTok{"label"}\NormalTok{, }\DecValTok{12}\NormalTok{, }\DecValTok{12}\NormalTok{, }\AttributeTok{label=}\StringTok{"On average, 89.8\% of users }
\StringTok{           wore their Fitbits on any given day."}\NormalTok{, }\AttributeTok{fill=}\StringTok{"white"}\NormalTok{) }\SpecialCharTok{+}
  \FunctionTok{labs}\NormalTok{(}\AttributeTok{title =} \StringTok{"Number of Users who wore their Fitbit on each day"}\NormalTok{,}
       \AttributeTok{x =} \StringTok{"Date"}\NormalTok{,}
       \AttributeTok{y =} \StringTok{"Number of Users"}\NormalTok{) }\SpecialCharTok{+}
  \FunctionTok{theme}\NormalTok{(}\AttributeTok{axis.text.x =} \FunctionTok{element\_text}\NormalTok{(}\AttributeTok{angle =} \DecValTok{90}\NormalTok{))}
\end{Highlighting}
\end{Shaded}

\includegraphics{Bellabeat-Case-Study_files/figure-latex/unnamed-chunk-8-1.pdf}

As the graph shows, an average of 89.8\% of users wore their Fitbit at
some point during the day. {This demonstrates that users are content to
use their devices during the day to track their activity.}

\hypertarget{how-many-users-wore-their-fitbit-device-to-bed-each-night}{%
\subsubsection{How many users wore their Fitbit device to bed each
night?}\label{how-many-users-wore-their-fitbit-device-to-bed-each-night}}

Next, I investigated how often users wore their devices at night.

\begin{Shaded}
\begin{Highlighting}[]
\CommentTok{\# Analysis of sleep data}

\FunctionTok{library}\NormalTok{(dplyr)}

\CommentTok{\# Find how many users out of 33 recorded their sleep at least once}
\NormalTok{num\_sleep\_users }\OtherTok{\textless{}{-}}\NormalTok{ sleepDay }\SpecialCharTok{\%\textgreater{}\%}
  \FunctionTok{summarise}\NormalTok{(}\AttributeTok{num\_unique =} \FunctionTok{n\_distinct}\NormalTok{(Id))}
\FunctionTok{print}\NormalTok{(num\_sleep\_users)}
\end{Highlighting}
\end{Shaded}

\begin{verbatim}
## # A tibble: 1 x 1
##   num_unique
##        <int>
## 1         24
\end{verbatim}

\begin{Shaded}
\begin{Highlighting}[]
\CommentTok{\# Find how many total sleep records were recorded by each user}
\NormalTok{total\_sleep\_records\_per\_id }\OtherTok{\textless{}{-}}\NormalTok{ sleepDay }\SpecialCharTok{\%\textgreater{}\%}
  \FunctionTok{group\_by}\NormalTok{(Id) }\SpecialCharTok{\%\textgreater{}\%}
  \FunctionTok{summarise}\NormalTok{(}\AttributeTok{total\_sleep\_records =} \FunctionTok{sum}\NormalTok{(TotalSleepRecords))}

\CommentTok{\# Create the bar plot}
\FunctionTok{ggplot}\NormalTok{(total\_sleep\_records\_per\_id, }\FunctionTok{aes}\NormalTok{(}\AttributeTok{x=}\FunctionTok{reorder}\NormalTok{(Id, total\_sleep\_records))) }\SpecialCharTok{+}
  \FunctionTok{geom\_bar}\NormalTok{(}\AttributeTok{mapping=}\FunctionTok{aes}\NormalTok{(}\AttributeTok{y=}\NormalTok{total\_sleep\_records), }\AttributeTok{stat =} \StringTok{"identity"}\NormalTok{, }\AttributeTok{fill =} \StringTok{"purple"}\NormalTok{, }\AttributeTok{alpha =} \FloatTok{0.7}\NormalTok{) }\SpecialCharTok{+} 
  \FunctionTok{annotate}\NormalTok{(}\AttributeTok{geom=}\StringTok{"label"}\NormalTok{, }\DecValTok{18}\NormalTok{, }\DecValTok{12}\NormalTok{, }\AttributeTok{label=}\StringTok{"Only 15/33 users recorded more than}
\StringTok{           10 sleep sessions."}\NormalTok{, }\AttributeTok{fill=}\StringTok{"white"}\NormalTok{) }\SpecialCharTok{+}
  \FunctionTok{labs}\NormalTok{(}\AttributeTok{title =} \StringTok{"Number of sleep records for each user"}\NormalTok{,}
       \AttributeTok{x =} \StringTok{"User ID"}\NormalTok{,}
       \AttributeTok{y =} \StringTok{"Number of Sleep Records"}\NormalTok{) }\SpecialCharTok{+}
  \FunctionTok{theme}\NormalTok{(}\AttributeTok{axis.text.x =} \FunctionTok{element\_text}\NormalTok{(}\AttributeTok{angle =} \DecValTok{90}\NormalTok{))}
\end{Highlighting}
\end{Shaded}

\includegraphics{Bellabeat-Case-Study_files/figure-latex/unnamed-chunk-9-1.pdf}

I found that only 24/33 users recorded sleep, and of those, 9/24
recorded less than 10 sleep records during the 31-day period. This gets
even worse when considering that some nights include multiple sleep
records, so for several users, the number of nights they wore the
tracker to bed is even less than the number of records. These are low
numbers, suggesting many people don't want to wear their tracker to bed
due to discomfort.

\hypertarget{how-many-users-used-the-weight-tracking-and-how-many-did-it-manually}{%
\subsubsection{How many users used the weight-tracking, and how many did
it
manually?}\label{how-many-users-used-the-weight-tracking-and-how-many-did-it-manually}}

First, I calculated the percentage of weighs that were done manually.

\begin{Shaded}
\begin{Highlighting}[]
\FunctionTok{library}\NormalTok{(tidyverse)}
\FunctionTok{library}\NormalTok{(dplyr)}

\CommentTok{\# Find the proportion of weight logs that are done manually}
\NormalTok{num\_manual\_weighs }\OtherTok{\textless{}{-}} \FunctionTok{sum}\NormalTok{(weightLogInfo}\SpecialCharTok{$}\NormalTok{IsManualReport }\SpecialCharTok{==} \ConstantTok{TRUE}\NormalTok{, }\AttributeTok{na.rm =} \ConstantTok{TRUE}\NormalTok{)}
\NormalTok{percent\_manual\_weighs }\OtherTok{\textless{}{-}}\NormalTok{ (num\_manual\_weighs}\SpecialCharTok{/}\DecValTok{67}\NormalTok{)}\SpecialCharTok{*}\DecValTok{100}
\FunctionTok{print}\NormalTok{(percent\_manual\_weighs)}
\end{Highlighting}
\end{Shaded}

\begin{verbatim}
## [1] 61.19403
\end{verbatim}

I found that \textasciitilde61\% of weighs are done manually, which
suggests that manual weight tracking wasn't a problem for most people
who used the weight-tracking feature. Additionally, 65/67 weight entries
had no fat value, suggesting two possibilities: a) Most people used
scales that didn't track fat composition, or b) They were using a scale
that tracked fat but didn't sync to the Fitbit app automatically, so
they were entering data manually and either forgot to or decided not to
add fat values to their entry.

\begin{Shaded}
\begin{Highlighting}[]
\CommentTok{\# Find how many total weigh{-}ins were done by each user}
\NormalTok{weigh\_id\_counts }\OtherTok{\textless{}{-}} \FunctionTok{table}\NormalTok{(weightLogInfo}\SpecialCharTok{$}\NormalTok{Id)}
\NormalTok{weigh\_id\_counts }\OtherTok{\textless{}{-}} \FunctionTok{as.data.frame}\NormalTok{(weigh\_id\_counts)}
\end{Highlighting}
\end{Shaded}

\begin{Shaded}
\begin{Highlighting}[]
\CommentTok{\# Create the bar plot}
\FunctionTok{ggplot}\NormalTok{(weigh\_id\_counts, }\FunctionTok{aes}\NormalTok{(}\AttributeTok{x=}\FunctionTok{reorder}\NormalTok{(Var1, Freq))) }\SpecialCharTok{+}
  \FunctionTok{geom\_bar}\NormalTok{(}\AttributeTok{mapping=}\FunctionTok{aes}\NormalTok{(}\AttributeTok{y=}\NormalTok{Freq), }\AttributeTok{stat =} \StringTok{"identity"}\NormalTok{, }\AttributeTok{fill =} \StringTok{"green"}\NormalTok{, }\AttributeTok{alpha =} \FloatTok{0.7}\NormalTok{) }\SpecialCharTok{+} 
  \FunctionTok{annotate}\NormalTok{(}\AttributeTok{geom=}\StringTok{"label"}\NormalTok{, }\DecValTok{3}\NormalTok{, }\DecValTok{15}\NormalTok{, }\AttributeTok{label=}\StringTok{"Only 8/33 users recorded their weight."}\NormalTok{, }\AttributeTok{fill=}\StringTok{"white"}\NormalTok{) }\SpecialCharTok{+}
  \FunctionTok{labs}\NormalTok{(}\AttributeTok{title =} \StringTok{"Number of weight recordings per user"}\NormalTok{,}
       \AttributeTok{x =} \StringTok{"User ID"}\NormalTok{,}
       \AttributeTok{y =} \StringTok{"Number of Weight Records"}\NormalTok{) }\SpecialCharTok{+}
  \FunctionTok{theme}\NormalTok{(}\AttributeTok{axis.text.x =} \FunctionTok{element\_text}\NormalTok{(}\AttributeTok{angle =} \DecValTok{90}\NormalTok{))}
\end{Highlighting}
\end{Shaded}

\includegraphics{https://i.imgur.com/aDJkJrs.png}

Through further analysis, I found out that only 8/33 users recorded any
weight data at all in Fitbit, and of those who did, 75\% had less than
10 records. Clearly, manual tracking was fine for people who used it,
but the problem is that hardly anyone did overall.

Considering how important it is to track weight, I found it strange that
so few users did so with their devices. It may suggest lack of access to
a convenient scale to use. {I believe that a convenient smart scale that
auto-syncs with the Bellabeat app could be a great new product for
Bellabeat to consider introducing to its customers.}

\hypertarget{conclusion}{%
\section{Conclusion}\label{conclusion}}

It is abundantly clear that users liked wearing their devices during the
day, as \textasciitilde90\% of them did on the average day. However,
usage dropped off significantly at night, and considering how important
sleep quality is for overall health, this is a problem. I think this is
because the devices are not comfortable enough for people to want to
wear them to bed. {Thus, comfort should be a top priority for Bellabeat
when designing its own wearable devices for women's health. They should
then run an ad campaign emphasizing the importance of quality sleep, and
highlight how women can best achieve it with new-and-improved,
ultra-comfortable Bellabeat trackers.}

Additionally, despite weight being a very important metric when tracking
one's own health, weight tracking was barely utilized by the 33 users
during this 31-day period. Less than 25\% of users (8/33) even recorded
their weight a single time, and only 25\% of those (2/8) had more than
10 entries. Also, the vast majority of weight entries had no fat values,
an important piece of sub-data relevant to weight. {This seems like an
issue of convenience, so I believe Bellabeat should design and market a
smart scale that tracks fat content and automatically syncs with their
app.}

Ultimately, the key insight here is that users want a comfortable device
that will passively collect all the health stats they care about. Thus,
comfort and convenience should be the \#1 priority in Bellabeat's
marketing strategy.

\end{document}
